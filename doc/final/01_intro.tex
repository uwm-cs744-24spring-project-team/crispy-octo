\section{Introduction}
% In recent years, the evolution of cloud services has revolutionized the landscape of of cluster provider's infrastructure~\cite{}. To accommodate various size of business and individuals, substantial works~\cite{} recently ensure the flexibility and scalability of cluster provider. Concurrently, the rise of microservices ~\cite{} among many companies such that the services require web applications, continuous integration and continuous delivery (CI/CD), reshape the infrastructure and open up new possibilities to optimize.


% One of the popular framework to initiate microservices or cloud native application is kubernetes ~\cite{}, due to the flexible and containerized structure. The containerized structure excels automated deployment, scaling and management. Kubernetes offers various capabilities, such as built-in service discovery, load balancing, fault tolerance, auto-scaling, etc. Despite these advancements, however, a discrepancy between resource allocation and utilization still notable. This gap arises from the disparity between client demands, typically driven by estimate peak usage, and the long-term resource requirements of certain services. As clients lack full control over user behavior over time, the utilization of allocated resources fluctuate over the life span of the application across a period of time. Use the web server as an example that demonstrates the characteristic that the requests from users may peak at midnight and gradually fall in the morning. In the daytime, low utilization of the requested resources contribute the majority of waste. 

% Emerging frameworks ~\cite{} address this challenge by overcommitment, a successful concept in many business model. Overcommitment, in essence, involves allocating resources beyond their physical capacity, leveraging statistical multiplexing and dynamic resource management to optimize utilization. This approach draws inspiration from industries where resources are constrained or demand is highly variable, such as airlines overselling seats or hotels oversubscribing rooms. By applying similar principles to computing environments, overcommitment frameworks aim to maximize resource utilization without compromising performance or reliability.

% In our project, we began by setting up an environment on Google Kubernetes Engine(GKE) and implementing a Koordinator ~\cite{} to manage overcommitment strategies. Leveraging Kubernetes's capabiulities for container orchestration and resource management, we configured the environment to dynamically colocating distinct workloads. With the infrastructure in place, we proceeded to evaluate the effectiveness of a colocating strategy by simulating workloads by Looksbusy ~\cite{}, then submitting Spark batch jobs to the Kubernetes cluster. Through our testing and analysis, we assessed the impact of utilizing such strategy in terms of resource utilization, performance, and scalability. By comparing the result against the Kubernetes' resource management, this work provide evidence that overcommitment and colocating strategies can significantly enhance resource utilization and efficiency.

In recent years, the evolution of cloud services has revolutionized the landscape of of cluster provider's infrastructure~\cite{noauthor_trends_2023}. To accommodate various size of business and individuals, substantial works~\cite{volcano}\cite{hindman2011mesos}\cite{ghodsi_dominant_nodate}\cite{narayanan_heterogeneity-aware_2020}\cite{romero2021infaas} recently ensure the flexibility and scalability of cluster provider.  

In this project, we plan to analyze different schedulers' behavior when scheduling low priority workloads (e.g. machine learning training jobs, daily conclusion data analysis jobs, etc.)  and latency-critic workloads (e.g. web servers, machine learning inference and real-time interactive data analysis jobs) together in single cluster.

In previous works, most schedulers only schedule single type of workloads:
In Mesos, each workload has its own scheduler \cite{hindman2011mesos}, and INFaaS \cite{romero2021infaas} only considers machine learning inference workloads, etc. But we might achieve better resource utilization by having a centralized scheduler to manage various types of workloads. 


Emerging frameworks ~\cite{koo}\cite{:aa}\cite{:kube-batch} address this challenge by over-commitment, a successful concept in many business model. Over-commitment, in essence, involves allocating resources beyond their physical capacity, leveraging statistical multiplexing and dynamic resource management to optimize utilization. This approach draws inspiration from industries where resources are constrained or demand is highly variable, such as airlines overselling seats or hotels oversubscribing rooms. By applying similar principles to computing environments, over-commitment frameworks aim to maximize resource utilization without compromising performance or reliability.

By co-locating batch jobs and real-time data analysis, we may further increase data locality among different jobs which manipulate the same partition of data, and reduce the cost of data transfer. This is especially useful in cloud data center scenario, because it can significantly reduce the overall cost for running a data center.

We plan to use Koordinator \cite{koo} as our scheduler, and compare its performance with default K8s scheduler. We began by setting up an environment on Google Kubernetes Engine(GKE) and insert a Koordinator ~\cite{koo} to manage overcommitment strategies. Leveraging Kubernetes's capabiulities for container orchestration and resource management, we configured the environment to dynamically colocating distinct workloads. With the infrastructure in place, we proceeded to evaluate the effectiveness of a colocating strategy by simulating workloads by Looksbusy ~\cite{looksbusy}, then submitting Spark batch jobs to the Kubernetes cluster. Through our testing and analysis, we assessed the impact of utilizing such strategy in terms of resource utilization, performance, and scalability. By comparing the result against the Kubernetes' resource management, this work provide evidence that over-commitment and colocating strategies can significantly enhance resource utilization and efficiency.